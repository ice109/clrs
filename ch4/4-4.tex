%% LyX 2.1.4 created this file.  For more info, see http://www.lyx.org/.
%% Do not edit unless you really know what you are doing.
\documentclass[oneside,english]{amsart}
\usepackage[T1]{fontenc}
\usepackage[latin9]{inputenc}
\usepackage{geometry}
\geometry{verbose}
\usepackage{amstext}
\usepackage{amsthm}
\usepackage{cancel}

\makeatletter
%%%%%%%%%%%%%%%%%%%%%%%%%%%%%% Textclass specific LaTeX commands.
\numberwithin{equation}{section}
\numberwithin{figure}{section}
\theoremstyle{plain}
\newtheorem{thm}{\protect\theoremname}
  \theoremstyle{definition}
  \newtheorem{problem}[thm]{\protect\problemname}

\makeatother

\usepackage{babel}
  \providecommand{\problemname}{Problem}
\providecommand{\theoremname}{Theorem}

\begin{document}
\begin{problem}
[4-4a]

\[
F\left(z\right)=\sum_{n=0}^{\infty}F_{n}z^{n}
\]
Let $F\left(z\right)=\left(0,1,1,2,3,5,8,13,\dots\right)$ the coefficients
of the monomials. Then multiplication by $z$
\[
zF\left(z\right)=\left(0,0,1,1,2,3,5,8,\dots\right)
\]
and
\[
z^{2}F\left(z\right)=\left(0,0,0,1,1,2,3,5,\dots\right)
\]
Hence 
\begin{alignat*}{2}
\quad z & = & \left(0,1,0,0,0,0,0,0,\dots\right)\;\\
\quad zF\left(z\right) & = & \left(0,0,1,1,2,3,5,8,\dots\right)\;\\
\underline{+z^{2}F\left(z\right)} & = & \left(0,0,0,1,1,2,3,5,\dots\right)\;\\
\quad F\left(z\right) & = & \quad\left(0,1,1,2,3,5,8,13,\dots\right)
\end{alignat*}

\end{problem}

\begin{problem}
[4.4b]

Since
\[
F\left(z\right)=z+zF\left(z\right)+z^{2}F\left(z\right)
\]
we have that 
\[
F\left(z\right)\left(1-z-z^{2}\right)=z
\]
or
\[
F\left(z\right)=\frac{z}{1-z-z^{2}}
\]
Factoring the denominator
\begin{eqnarray*}
F\left(z\right) & = & \frac{z}{-\left(z+\frac{\left(1-\sqrt{5}\right)}{2}\right)\left(z+\frac{1+\sqrt{5}}{2}\right)}\\
 & = & \frac{z}{\left(1-z\left(\frac{1+\sqrt{5}}{2}\right)\right)\left(1-z\left(\frac{1-\sqrt{5}}{2}\right)\right)}\\
 & = & \frac{z}{\left(1-\phi z\right)\left(1-\hat{\phi}z\right)}\\
 & = & \frac{1}{\sqrt{5}}\left(\frac{1}{\left(1-\phi z\right)}-\frac{1}{\left(1-\hat{\phi}z\right)}\right)
\end{eqnarray*}

\end{problem}

\begin{problem}
[4.4c] 

Using the Taylor series 
\[
\frac{1}{1-x}=\sum_{n=0}^{\infty}x^{n}
\]
we have by above
\begin{eqnarray*}
F\left(z\right) & = & \frac{1}{\sqrt{5}}\left(\frac{1}{\left(1-\phi z\right)}-\frac{1}{\left(1-\hat{\phi}z\right)}\right)\\
 & = & \frac{1}{\sqrt{5}}\left(\sum_{n=0}^{\infty}\left(\phi z\right)^{n}-\sum_{n=0}^{\infty}\left(\hat{\phi}z\right)^{n}\right)\\
 & = & \sum_{n=0}^{\infty}\frac{1}{\sqrt{5}}\left(\phi^{n}-\hat{\phi}^{n}\right)z^{n}
\end{eqnarray*}

\end{problem}

\begin{problem}
[4.4d] 

By comparing coefficients in the the original generating function
and the re-expression
\[
F\left(z\right)=\sum_{n=0}^{\infty}F_{n}z^{n}=\sum_{n=0}^{\infty}\frac{1}{\sqrt{5}}\left(\phi^{n}-\hat{\phi}^{n}\right)z^{n}
\]
we see that 
\[
F_{n}=\frac{1}{\sqrt{5}}\left(\phi^{n}-\hat{\phi}^{n}\right)
\]
Since $\left|\hat{\phi}\right|<1$ it's the case that $\left|\hat{\phi}^{n}\right|<1$
and hence is fractional.
\end{problem}

\section*{Exponentiation by Squaring}

Note that 
\[
x^{n}=\begin{cases}
x\left(x^{2}\right)^{\frac{n-1}{2}} & \text{if }n\text{ is odd}\\
\left(x^{2}\right)^{\frac{n}{2}} & \text{if }n\text{ is even}
\end{cases}
\]
This is recursive. For example for $x^{10}$:

\begin{eqnarray*}
x^{20} & = & \left(x^{2}\right)^{20/2}=\left(x^{2}\right)^{10}\\
\left(x^{2}\right)^{10} & = & \left(x^{2}\times x^{2}\right)^{10/2}=\left(x^{2}\times x^{2}\right)^{5}\\
\left(x^{2}\times x^{2}\right)^{5} & = & x^{4}\left(\left(x^{2}\times x^{2}\right)\times\left(x^{2}\times x^{2}\right)\right)^{\frac{5-1}{2}}=x^{4}\left(\left(x^{2}\times x^{2}\right)\times\left(x^{2}\times x^{2}\right)\right)^{2}\\
\left(\left(x^{2}\times x^{2}\right)\times\left(x^{2}\times x^{2}\right)\right)^{2} & = & \left(\left[\left(x^{2}\times x^{2}\right)\times\left(x^{2}\times x^{2}\right)\right]\left[\left(x^{2}\times x^{2}\right)\times\left(x^{2}\times x^{2}\right)\right]\right)
\end{eqnarray*}
i.e 
\begin{eqnarray*}
x^{20} & = & \left(x^{2}\right)^{10}\\
 & = & \left(x^{2}x^{2}\right)^{5}\\
 & = & \left(x^{2}x^{2}\right)^{1}\left(x^{2}x^{2}\right)^{4}\\
 & = & \left(x^{2}x^{2}\right)\left(\left(x^{2}x^{2}\right)\left(x^{2}x^{2}\right)\right)^{2}\\
 &  & \left(x^{2}x^{2}\right)\left(\left[\left(x^{2}x^{2}\right)\left(x^{2}x^{2}\right)\right]\left[\left(x^{2}x^{2}\right)\left(x^{2}x^{2}\right)\right]\right)
\end{eqnarray*}
Bottom up though for $n=20$ we have
\begin{eqnarray*}
 &  & y_{1}=1,n=20\\
x_{1}=x\times x & = & x^{2},n=20/2=10\\
x_{2}=\left(x_{1}\right)^{2}=x^{2}\times x^{2} & = & x^{4},n=10/2=5\\
y_{2}=x_{2}=x^{4},x_{3}=\left(x_{2}\right)^{2}=x^{4}\times x^{4} & = & x^{8},n=\left(5-1\right)/2=2\\
x_{4}=\left(x_{3}\right)^{2}=x^{8}\times x^{8} & = & x^{16},n=2/2=1
\end{eqnarray*}
You ``slide out'' the +1 of an odd number.
\end{document}
